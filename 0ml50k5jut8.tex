\begin{equation}
h^f() = h^f_{ref} -(T-T_{ref})c_p^f \approx 0 - 4.186*  C^\circ   
\end{equation}
where \(h_{ref}^f=0\) and \(T_{ref} = 273.16\, K^\circ\) so that temperature is now measured in centigrades. Assume also \(C^\circ >0 \).
At the same time for \(g\) is 
\begin{equation}
h^g( ) \approx 2.26  - 1.996 C^\circ
\end{equation}
and therefore:
\begin{equation}
h^g()-h^f() \approx 2.26 - 2.19 C^\circ \, {\rm if}\ C^\circ > 0
\end{equation}
For what regards \( c_p\), we have:
\begin{equation}
c_p() \approx c_p^e \theta^e + 4.186 \theta^f + 1.996 \theta^g
\end{equation}
and is, in any case, positive.
Therefore from equations (15), (19) and (20):
\begin{equation}
(c_p^e \theta^e + 4.186 \theta^f + 1.996 \theta^g) \frac{dT}{dt} = 2.26 - 2.19 C^\circ \frac{d\theta^f}{dt}\, {\rm if}\ C^\circ > 0
\end{equation}
This wold imply that not for any temperature and water fraction variation we have cooling. The condition for cooling, if I  is that:
\begin{equation}
\end{equation}